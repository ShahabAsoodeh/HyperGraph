\def\year{2019}\relax
%File: formatting-instruction.tex
\documentclass[letterpaper]{article} %DO NOT CHANGE THIS
%\newcommand{\CLASSINPUTbaselinestretch}{1.2}
%\documentclass[conference,twocolumn]{IEEEtran}
%\documentclass[letterpaper,10 pt,conference]{ieeeconf} 
%\documentclass[10pt, onecolumn]{IEEEtran}
%\documentclass[10pt,letterpaper]{article}
%\IEEEoverridecommandlockouts    
% \overrideIEEEmargins


%\usepackage[left=1in,top=1in,right=1in,bottom=1in]{geometry}


% \usepackage{geometry}
% \geometry{
% a4paper,
% total={210mm,297mm},
% left=17.1mm,
% right=17.1mm,
% top=19.1mm,
% bottom=17.1mm,
% }


\usepackage[english]{babel}
\usepackage{enumitem, hyperref}
\usepackage[latin1]{inputenc}
\usepackage{enumerate}
%\usepackage{enumitem}
%\setlist[enumerate]{label=\roman*}  %% for all enumerate environments
\usepackage{graphicx}
\usepackage{color}
\usepackage[dvipsnames]{xcolor}
\usepackage[T1]{fontenc}
\usepackage{subfigure}
%\usepackage{subcaption}
\usepackage{dsfont}
\usepackage{amsfonts}
\usepackage{graphicx,wrapfig}
\usepackage[T1]{fontenc}
\usepackage{amsmath}
\usepackage{mathtools, cuted}
\usepackage{amsthm}
\usepackage{amstext}
\usepackage{amssymb}
\usepackage{mathrsfs}
%\usepackage{cite}
\usepackage{mathtools}
\usepackage{tikz}
\usepackage{marginnote}
\usepackage{tkz-tab}
\usetikzlibrary{topaths,calc}
%*****Margin Note
\usepackage{stackengine}

%**********

\newcommand{\appsec}{
	\renewcommand{\thesubsection}{\Alph{subsection}}
}
\def\R{\mathbb{R}}
\def\cE{\mathcal{E}}
\def\fin{{<\infty}}
\def\supp{\mathop{\rm supp}}
\def\Gr{\mathop{\rm Gr}}
\def\cl{\mathop{\rm cl}}
\def\Proj{\mathop{\rm Proj}}
\def\argmax{\mathop{\rm arg\, max}}
\def\argmin{\mathop{\rm arg\, min}}
\def\eps{\varepsilon}
\def\hX{\hat{X}}
\def\hY{\hat{Y}}
\def\hx{\hat{x}}
\def\hf{\hat{f}}
\def\A{{\mathcal A}}
\def\B{{\mathcal B}}
\def\D{{\mathcal D}}
\def\F{{\mathcal F}}
\def\H{{\mathcal H}}
\def\I{{\mathcal I}}
\def\G{{\mathcal G}}
\def\C{{\mathcal C}}
\def\P{{\mathcal P}}
\def\Q{{\mathcal Q}}
\def\T{{\mathcal T}}
\def\X{{\mathcal X}}
\def\Y{{\mathcal Y}}
\def\M{{\mathcal M}}
\def\N{{\mathcal N}}
\def\S{{\mathcal S}}
\def\L{{\mathcal L}}
\def\cR{{\mathcal R}}
\def\Z{{\mathcal Z}}
\def\U{{\mathcal U}}
\def\V{{\mathcal V}}
%\newcommand{\indep}{\perp\!\!\!\perp}
\def\indep{{\perp\!\!\!\perp}}
\def\sgr{{\mathsf{gr}}}
\def\spr{{\mathsf{pr}}}
\def\sPr{{\mathsf{Pr}}}
\def\sUnif{{\mathsf{Unif}}}
\def\sBer{{\mathsf{Bernoulli}}}
\def\sP  {{\mathsf{P}}}
\def\sp{{\mathsf p}}
\def\sq{{\mathsf q}}
\def\sX{{\mathsf X}}
\def\sY{{\mathsf Y}}
\def\sZ{{\mathsf Z}}
\def\sM{{\mathsf {sENSR}}}
\def\sE{{\mathsf E}}
\def\sF{{\mathsf F}}
\def\sU{{\mathsf U}}
\def\sV{{\mathsf V}}
\def\sW{{\mathsf {W}}}
\def\sG{{\mathsf G}}
\def \var {{\mathsf {var}   }}
\def \mmse {{\mathsf {mmse}   }}
\def \cP {\mathsf{P}_{\mathsf{c}}}
\newcommand{\dsty}[1]{$\displaystyle #1$}
\def \AI {{I^{\mathsf{A}}}}
\def \nP {{\mathsf{P}^{(n)}}}
\def \np {{\mathsf{p}^{(n)}}}
\def \nq {{\mathsf{q}^{(n)}}}
\newcommand{\repdc}[3]{#1_{#2} , \ldots , #1_{#3}}
\usepackage[font=small,labelsep=space]{caption}
\captionsetup{%
	figurename=Fig.,
}
%\captionsetup{labelsep = }
\DeclareCaptionLabelSeparator{dot}{.~}
%\DeclareCaptionLabelSeparator{bar}{ | }
\captionsetup{
	labelsep=dot
}
%\theoremstyle{theorem}
%\theoremstyle{example}
\newcounter{example}
\newenvironment{example}[1][]{\refstepcounter{example}\par\medskip
	\noindent \textit{Example~\theexample. #1} \rmfamily}{\medskip}
%\newtheorem{example}{Example}
\newtheorem{definition}{Definition}
\newtheorem{theorem}{Theorem}
\newtheorem{corollary}{Corollary}
\newtheorem{fact}{Fact}
\newtheorem{proposition}{Proposition}
\newtheorem{lemma}{Lemma}
\theoremstyle{remark}
\newtheorem{remark}{Remark}
\newtheorem{conjecture}{Conjecture}
\newcommand{\markov}{\mathrel\multimap\joinrel\mathrel-%
	\mspace{-9mu}\joinrel\mathrel-}

\renewcommand{\qedsymbol}{\rule{0.5em}{0.5em}}
%%%%%%%%%%%%%%%%%%%%%%%%%%%%%%%%%%%%%%%%%%%%%%%%%%%%%%%%%%%%%%%%%%%%%%%%%%%%%%%%%%%%%%%%%55

\usepackage{times}
\usepackage{tikz}
\usepackage{amsmath}
\usepackage{verbatim}
\usetikzlibrary{arrows,shapes}
\tikzstyle{RectObject}=[rectangle,fill=white,draw,line width=0.2mm]
\tikzstyle{line}=[draw]
\tikzstyle{arrow}=[draw, -latex]
\usetikzlibrary{decorations.pathmorphing}
\usetikzlibrary{calc,shapes, positioning}
\usepackage{graphicx}
\usepackage{caption}
\usetikzlibrary{shapes.geometric}


\definecolor{DukeBlue}{HTML}{001A57}
\definecolor{DarkRed}{rgb}{0.75, 0.0, 0.0}
\definecolor{DarkGreen}{rgb}{0.0, 0.5, 0.0}
\newcommand{\TG}[1]{\textbf{\textcolor{DukeBlue}{(#1)}}}
\newcommand{\SK}[1]{\textbf{\textcolor{Magenta}{[#1]}}}
\newcommand{\md}[1]{\textcolor{Magenta}{#1}}



\DeclareFontFamily{U}{BOONDOX-calo}{\skewchar\font=45 }
\DeclareFontShape{U}{BOONDOX-calo}{m}{n}{
  <-> s*[1.05] BOONDOX-r-calo}{}
\DeclareFontShape{U}{BOONDOX-calo}{b}{n}{
  <-> s*[1.05] BOONDOX-b-calo}{}
\DeclareMathAlphabet{\mathcalboondox}{U}{BOONDOX-calo}{m}{n}
\SetMathAlphabet{\mathcalboondox}{bold}{U}{BOONDOX-calo}{b}{n}
\DeclareMathAlphabet{\mathbcalboondox}{U}{BOONDOX-calo}{b}{n}

%%%%%%%%%%%%%%%%%%%%%%%%%%%%%%%%%%%%%%%%%%%%%%%%%%%%%%%%%%%%%%%%%%%%%%%%%%%%%%%%%%%%%%%%5

% \IEEEoverridecommandlockouts

\allowdisplaybreaks

% Yi's delimiter
\newcommand{\paren}[1]{\left(#1\right)}
\newcommand{\set}[1]{\left\{#1\right\}}

\usepackage{aaai19}  %Required
\usepackage{times}  %Required
\usepackage{helvet}  %Required
\usepackage{courier}  %Required
\usepackage{url}  %Required
\usepackage{graphicx}  %Required
\frenchspacing  %Required
\setlength{\pdfpagewidth}{8.5in}  %Required
\setlength{\pdfpageheight}{11in}  %Required

%PDF Info Is Required:
\pdfinfo{
/Title (Supplementary Materials for "Wasserstein Soft Label Propagation on Hypergraphs: Algorithm and Generalization Error Bounds")
/Author (AAAI Press Staff)}
\setcounter{secnumdepth}{0}  

\begin{document}
% The file aaai.sty is the style file for AAAI Press 
% proceedings, working notes, and technical reports.
\title{Supplemental Material for "Wasserstein Soft Label Propagation on Hypergraphs: Algorithm and Generalization Error Bounds"}
\author{AAAI Press\\
Association for the Advancement of Artificial Intelligence\\
2275 East Bayshore Road, Suite 160\\
Palo Alto, California 94303\\
}
\maketitle

\begin{abstract}
Below are the supplemental material for the paper  "Wasserstein Soft Label Propagation on Hypergraphs: Algorithm and Generalization Error Bounds"   submitted to AAAI 2019 
\end{abstract}
	
\section{Proof of Lemma~\ref{lem:maximum-principle}}
  The conditions on $\Phi$ can be written as
  \begin{align}
      \left[\frac{t_i}{m\gamma}+\mathrm{deg}\left(i\right)\right]\Phi \left( i \right)-\sum_{j:j\sim i}\Phi \left( j \right)\geq 0 & \qquad 1\leq i\leq \ell\label{eq:maximum-principle-boundary}\\
      \mathrm{deg}\left(i\right)\Phi \left( i \right)-\sum_{j:j\sim i}\Phi \left( j \right)=0 & \qquad \ell+1\leq i\leq n\label{eq:maximum-principle-interior}
  \end{align}
  where $\mathrm{deg}\left( i \right)\geq 1$ is the degree of vertex $i$ in graph $G$. First, we assert that the minimum of $\Phi$ must be attained among the vertices $1,\cdots,\ell$, for otherwise, if $\ell+1\leq i_*=\argmin_{i\in V}\Phi \left( i \right)\leq n$, then by \eqref{eq:maximum-principle-interior} we have
  \begin{equation*}
    \begin{aligned}
      \mathrm{deg}\left(i_*\right)\Phi \left( i_* \right)&=\sum_{j:j\sim i_*}\Phi \left( j \right)\\
      &\geq \sum_{j:j\sim i_*}\Phi \left( i_* \right)=\mathrm{deg}\left(i_*\right)\Phi \left( i_* \right)
    \end{aligned}
  \end{equation*}
which implies $\Phi \left( j \right)=\Phi \left( i_* \right)$ for all vertices $j\sim i_*$. This argument can be repeated until the constant value propagates into the vertices within $1,\cdots,\ell$, and the assertion follows from the connectivity of the graph. The assertion for the maximum can be established analogously. Next we argue that the minimum of $\Phi$ on the vertices of $G$ must be non-negative. Assume the contracy, i.e. the minimum attained at $i_*\in \left[ 1,\ell \right]$ is strictly negative, then by \eqref{eq:maximum-principle-boundary} we have
\begin{equation*}
  \begin{aligned}
    0 &\leq \left[\frac{t_{i_*}}{m\gamma}+\mathrm{deg}\left(i_{*}\right)\right]\Phi \left( i_{*} \right)-\sum_{j:j\sim i_{*}}\Phi \left( j \right)\\
    &=\frac{t_{i_*}}{m\gamma}\Phi \left( i_* \right)+\sum_{j:j\sim i_*}\left[ \Phi \left( i_* \right)-\Phi \left( j \right) \right]<0
  \end{aligned}
\end{equation*}
 where the strict inequalty follows from the counter-assumption $\Phi \left( i_{*} \right)<0$. This contradiction completes our proof that $\Phi \geq 0$ on the entire graph $G$.

\section{Proof of Proposition~\ref{Propo:Non_Decreasing}}
  By the equivalence of \eqref{eq:tikhonov-one-dimensional-subproblem-equiv} and \eqref{eq:tikhonov-one-dimensional-subproblem}, the solutions $\Phi_s$ satisfies the Euler-Lagrange equations for \eqref{eq:tikhonov-one-dimensional-subproblem}:
  \begin{equation*}
    \left( T_{\ell}+m\gamma L \right)\Phi_s^{*}=\mathbf{y}_s.
  \end{equation*}
  For any $0\leq t\leq s\leq 1$, subtracting two Euler-Lagrange equations yields
  \begin{equation*}
    \left( T_{\ell}+m\gamma L \right)\left(\Phi_s^{*}-\Phi_t^{*}\right)=\mathbf{y}_s-\mathbf{y}_t\geq 0
  \end{equation*}
  where the inequality follows from the definition of $\mathbf{y}_s$ in \eqref{eq:y-defn}. Furthermore, it is straightforward to see that $\mathbf{y}_s-\mathbf{y}_t$ satisfies the assumption in Lemma~\ref{lem:maximum-principle}, which then implies $\Phi_s^{*}\geq \Phi_t^{*}$.

\section{Proof~\ref{thm:slice-algorithmic-stability}}
  Following the same argument as in the proof of \cite[Theorem 5]{Belkin2004}, we can assume without loss of generality that $S$, $S'$ differ by a new point $\left( v_m,\mu_m \right)\leftrightarrow \left( v_m',\mu_m' \right)$; the other case where only the multiplicities differ can be treated similarly. By our assumption \eqref{eq:quantile-boundedness}, the two averages differ by at most an amount of
  \begin{equation*}
    \left| \bar{y}_s-\bar{y}_s' \right|\leq \frac{2M_s}{m}.
  \end{equation*}
  For simplicity, introduce temporary notations
  \begin{equation*}
    A:=T_{\ell}+m\gamma L,\qquad B:=T_{\ell}'+m\gamma L.
  \end{equation*}
  Using the simple fact that the $2$-norm dominate the $\infty$-norm, we have
  \begin{equation*}
    \begin{aligned}
      &\left\| \Phi_s^{*} -\Phi_s'^{*} \right\|_{\infty} \leq \left\| \Phi_s^{*} -\Phi_s'^{*} \right\|_2\\
      &\leq \frac{2M_s}{m}+\left\| A^{-1}\left(\mathbf{y}_s-\bar{y}_sT_\ell\mathbf{1}\right)-B^{-1}\left(\mathbf{y'}_s-\bar{y}_s'T'_\ell\mathbf{1}\right) \right\|_2\\
      &\leq \frac{2M_s}{m}+\left\| A^{-1}\left(\mathbf{y}_s-\bar{y}_sT_\ell\mathbf{1}\right)-A^{-1}\left(\mathbf{y'}_s-\bar{y}_s'T'_\ell\mathbf{1}\right) \right\|_2\\
      &\qquad+\left\| A^{-1}\left(\mathbf{y'}_s-\bar{y}_s'T'_\ell\mathbf{1}\right)-B^{-1}\left(\mathbf{y'}_s-\bar{y}_s'T'_\ell\mathbf{1}\right) \right\|_2.
    \end{aligned}
  \end{equation*}
  Stanard functional analysis argument (the same perturbation reasoning we gave in \eqref{eq:standard-functional-analysis}) tells us that $\left\| A^{-1} \right\|_2\leq \left( m\gamma\lambda_1-T \right)^{-1}$. Together with the observation that
    \begin{align*}
      &\left\| \left(\mathbf{y}_s-\bar{y}_sT_\ell\mathbf{1}\right) - \left(\mathbf{y'}_s-\bar{y}_s'T'_\ell\mathbf{1}\right) \right\|_2\\
      &\leq \left\| \mathbf{y}_s-\mathbf{y}_s' \right\|_2+\left\| \bar{y}_sT_\ell\mathbf{1}-\bar{y}_s'T'_\ell\mathbf{1} \right\|_2\\
      &\leq 2M_s+\frac{2M_s}{m}<4M_s
    \end{align*}
  we have
  \begin{equation*}
    \left\| A^{-1}\left(\mathbf{y}_s-\bar{y}_sT_\ell\mathbf{1}\right)-A^{-1}\left(\mathbf{y'}_s-\bar{y}_s'T'_\ell\mathbf{1}\right) \right\|_2\leq \frac{4M_s}{m\gamma\lambda_1-T}.
  \end{equation*}
  In the meanwhile, noting that we also have $\left\| B^{-1} \right\|_2\leq \left( m\gamma\lambda_1-T \right)^{-1}$, and $\left\| A-B \right\|_2=\left\| T_{\ell}'-T_{\ell} \right\|_2 \leq \sqrt{2}<3/2$, we conclude that
  \begin{equation*}
    \begin{aligned}
      &\left\| A^{-1}\left(\mathbf{y'}_s-\bar{y}_s'T'_\ell\mathbf{1}\right)-B^{-1}\left(\mathbf{y'}_s-\bar{y}_s'T'_\ell\mathbf{1}\right) \right\|_2\\
      &=\left\| B^{-1} \left( B-A \right) A^{-1}\left(\mathbf{y'}_s-\bar{y}_s'T'_\ell\mathbf{1}\right)\right\|_2\leq \frac{3M_s\sqrt{Tm}}{\left( m\gamma\lambda_1-T \right)^2}.
    \end{aligned}
  \end{equation*}
  Putting everything together completes the proof.

\section{Proof of Lemma~\ref{lem:apriori-estimates}}

  By the equivalence between \eqref{eq:tikhonov} and \eqref{eq:tikhonov-one-dimensional}, it suffices to show the following fact: for each fixed $s\in \left[ 0,1 \right]$, if $\max \left\{ \left|F_{\mu_i}^{-1} \left( s \right)\right|,\,\,i=1,\cdots,m\right\}\leq \phi \left( s \right)$ then $\left\|\Phi_s^{*}\right\|_{\infty}\leq \phi \left( s \right)$, where $\Phi_s^{*}$ is defined in \eqref{eq:tikhonov-one-dimensional-subproblem-equiv}. But this follows straightforwardly from the maximum principle.

\section{Proof of Proposition~\ref{prop:alg-stability-slp}}
Let $\left( j,\theta_j \right)$ be a new sample drawn from the joint distribution $D$. Then $\theta_j\in\mathcal{M}_{\phi}^2$ with probability $1$. Let $S$, $S'$ be two training samples with values in $\mathcal{M}_{\phi}^2$ and differ by exactly one data point. By Theorem~\ref{thm:slice-algorithmic-stability} we have
  \begin{equation}
    \label{eq:sliced-bounds}
    \begin{aligned}
      &\left| \Phi_s^{*} \left( j \right) -\Phi_s'^{*} \left( j \right) \right| \\
      &\leq \left[\frac{3\sqrt{Tm}}{\left( m\gamma\lambda_1-T \right)^2}+\frac{4}{m\gamma\lambda_1-T}+\frac{2}{m}\right]\phi \left( s \right).
    \end{aligned}
  \end{equation}
%  By \eqref{eq:one-dimensional-wasserstein},
By \eqref{Eq:One_D_Equivalence}, the difference between the squared Wasserstein losses satisfy
  \begin{equation*}
    \begin{aligned}
      &\left| c \left( f_S, \left( j,\theta_j \right) \right) - c \left( f_{S'}, \left( j,\theta_j \right) \right) \right|\\
      &=\left| W_2^2 \left( f_S \left( j \right),\theta_j \right) - W_2^2 \left( f_{S'} \left( j \right),\theta_j \right) \right|\\
      &=\left| \int_0^1 \left| \Phi_s^{*} \left( j \right)-F_{\theta_j}^{-1}\left( s \right) \right|^2\mathrm{d}s - \int_0^1 \left| \Phi_s'^{*} \left( j \right)-F_{\theta_j}^{-1}\left( s \right) \right|^2\mathrm{d}s \right|\\
      &\leq \int_0^1 \left| \left( \Phi_s^{*} \left( j \right)+\Phi_s'^{*} \left( j \right)-2F_{\theta_j}^{-1}\left( s \right) \right)\left( \Phi_s^{*} \left( j \right) -\Phi_s'^{*} \left( j \right) \right) \right| \mathrm{d}s\\
      &\stackrel{\left( * \right)}{\leq} \left[\frac{3\sqrt{Tm}}{\left( m\gamma\lambda_1-T \right)^2}+\frac{4}{m\gamma\lambda_1-T}+\frac{2}{m}\right]\cdot \int_0^1 4\phi \left( s \right)\cdot \phi \left( s \right)\mathrm{d}s\\
      &=4\left\| \phi \right\|_2^2\left[\frac{3\sqrt{Tm}}{\left( m\gamma\lambda_1-T \right)^2}+\frac{4}{m\gamma\lambda_1-T}+\frac{2}{m}\right]=\beta,
    \end{aligned}
  \end{equation*}
  where at $\left( * \right)$ we used \eqref{eq:sliced-bounds} to bound the difference $\left| \Phi_s^{*} \left( j \right) -\Phi_s'^{*} \left( j \right) \right|$, and invoked Lemma~\ref{lem:apriori-estimates} to conclude that
  \begin{equation*}
    \Phi_s^{*}\left( j \right),\Phi_s'^{*}\left( j \right) \leq \phi \left( s \right)
  \end{equation*}
  and hence
  \begin{equation*}
    \left| \Phi_s^{*} \left( j \right)+\Phi_s'^{*} \left( j \right)-2F_{\theta_j}^{-1}\left( s \right) \right|\leq 4\phi \left( s \right).
  \end{equation*}

\section{Proof of Theorem~\ref{Thm:Gen_Error_Soft}}


  Note that the cost function is uniformly bounded by $M=4 \left\| \phi \right\|_2^2$ in our setting. The rest follows from Proposition~\ref{prop:alg-stability-slp} and Theorem~\ref{thm:bousquet-elisseeff}.






%\bibliography{bibliography}
%  \bibliographystyle{aaai}	

\end{document}